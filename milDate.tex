\documentclass{scrarticle}
\usepackage{fontspec}
\usepackage[lang=english]{milDate}
\usepackage{xcolor}

\usepackage{polyglossia}
\setmainlanguage[babelshorthands=true]{german}
\setotherlanguage[variant=british]{english}

\defaultfontfeatures{Scale=MatchLowercase}
\setmainfont[Ligatures=TeX]{Linux Libertine O}
\setsansfont[Ligatures=TeX]{Linux Biolinum O}

\definecolor{grass}{HTML}{38A808}
\definecolor{denim}{HTML}{0009A8}
\definecolor{sunset}{HTML}{F55D18}

\setlength{\parindent}{0pt}

\title{\textnormal{The \textbf{milDate} Package}}
\author{Alexander Bernardi}
%\date{\milDate{17.01.2023}}

\begin{document}
\maketitle

The \textbf{milDate} Package provides commands for displaying the date in NATO format.\par
The moon landing on July 21th 1969 at 02:56 is represented in the date time group as
\par\begin{center}
\fcolorbox{denim}{denim!5!white}{21}
\fcolorbox{sunset}{sunset!5!white}{02}
\fcolorbox{sunset}{sunset!5!white}{56}
\fcolorbox{grass}{grass!5!white}{Z}
\fcolorbox{denim}{denim!5!white}{jul}
\fcolorbox{denim}{denim!5!white}{69}
\end{center}\par

 21. Juli 1969 um 3:56 Uhr MEZ
\milDate{17-04-2012}\par
\milTime{12:10}[Z]\par
\milDatetime{17.01.2023}{14:00}
\end{document}