\documentclass[a4paper,10pt]{scrarticle}
\usepackage{fontspec}

\usepackage{polyglossia}
\setmainlanguage[variant=british]{english}
\setotherlanguage[babelshorthands=true]{german}

\usepackage{xcolor}
\usepackage{booktabs}
\usepackage{milDate}

\defaultfontfeatures{Scale=MatchLowercase}
\setmainfont[Ligatures=TeX]{Linux Libertine O}
\setsansfont[Ligatures=TeX]{Linux Biolinum O}

\definecolor{grass}{HTML}{38A808}
\definecolor{denim}{HTML}{0009A8}
\definecolor{sunset}{HTML}{F55D18}

\setlength{\parindent}{0pt}
\reversemarginpar

\newcommand\alphabet{abcdefghijklmnopqrstuvwxyz}
\newlength{\charwidth}
\settowidth{\charwidth}{\alphabet}
\setlength{\textwidth}{2.7\charwidth}

\title{\textnormal{The \textbf{milDate} Package}}
\author{Alexander Bernardi}
%\date{\milDate{17.01.2023}}

\begin{document}
\maketitle

\section{Introduction}

The \textbf{milDate} Package provides commands for displaying the date and time in NATO format.\par
The format is used to standardize the communication of dates.
A Date-Time-Group consists of a sequence of eight digits and four letters. The first two digits describe the day, the following four digits the time in 24-hour format, followed by a letter indicating the time zone. After the time zone the month is indicated by a uniform abbreviation consisting of three letters and at the end optionally follows the indication of the year in the form of two digits. \par
For example the moon landing on July 21th 1969 at 02:56 is represented in the date time group as
\par\begin{center}
\fcolorbox{denim}{denim!5!white}{21}
\fcolorbox{sunset}{sunset!5!white}{02}
\fcolorbox{sunset}{sunset!5!white}{56}
\fcolorbox{grass}{grass!5!white}{Z}
\fcolorbox{denim}{denim!5!white}{jul}
\fcolorbox{denim}{denim!5!white}{69}
\end{center}\par

\fcolorbox{denim}{denim!5!white}{21}
\fcolorbox{denim}{denim!5!white}{jul}
\fcolorbox{denim}{denim!5!white}{69} indicates the date, July 21th 1969\par\medskip

\fcolorbox{sunset}{sunset!5!white}{02}
\fcolorbox{sunset}{sunset!5!white}{56} indicates the time 02:56 with \par\medskip
\fcolorbox{grass}{grass!5!white}{Z} as the corresponding timezone Zulu\par\medskip

The date alone

\par\begin{center}
\fcolorbox{denim}{denim!5!white}{21}
\fcolorbox{denim}{denim!5!white}{jul}
\fcolorbox{denim}{denim!5!white}{69}
\end{center}\par

respectively the time

\par\begin{center}
\fcolorbox{sunset}{sunset!5!white}{02}
\fcolorbox{sunset}{sunset!5!white}{56}
\fcolorbox{grass}{grass!5!white}{Z}
\end{center}\par

\section{Options}

The \textbf{milDate} package can be customized by global options.

\verb+\usepackage[lang=english,timezone=Z,kern=1pt]{milDate}+

\noindent\marginpar{lang} The \textbf{milDate} package identifies the language used in the document and sets the month names accordingly. The behavior can be overridden by setting the \verb+lang+ option.
Currently german and english are available.

\noindent\marginpar{timezone} The \verb+timezone+ option sets the default time zone that will be used globally if no time zone is specified in the date time information. For the possible time zones, see the timezone table. Zulu time is used by default.

\noindent\marginpar{kern} The individual blocks of the date-time group are set with a small space between them. By default, the \textbf{milDate} package uses 1pt. This spacing can be changed with the \verb+kern+ option if needed. A \emph{kern} is a typographic term for a nonbreakable space between two elements.

\section{Commands}

\noindent\marginpar{milTime} The \textbf{milTime} command sets the time in 24 hour format hhmm without delimiter.

\par\medskip
\begin{tabular}{ll}
\verb+\milTime{12:30}[S]+ & \milTime{12:30}[S] \\	% JFK assassinated
\verb+\milTime{0238}+	  & \milTime{0238} \\		% Titanic hits iceberg
\end{tabular}

\medskip\noindent\marginpar{timezone} Time zones are indicated with letters A to I and K to Z. To avoid confusion, J is skipped. Starting from the prime meridian at Greenwich (UTC-0, Greenwich Mean Time \emph{GMT}, Zulu time), counting is eastward from A to M and westward from N to Y.

\bigskip\begin{minipage}{0.4\textwidth}\label{tab:timezones}
\begin{tabular}{rcl}
\toprule
UTC & & Example \\
\midrule
-12 & Y & Fiji \\
-11 & X & Samoa \\
-10 & W & Honolulu \\
-9 & V & Juneau \\
-8 & U & \emph{Los Angeles} \\
-7 & T & \emph{Denver} \\
-6 & S & \emph{Dallas} \\
-5 & R & \emph{New York} \\
-4 & Q & \emph{Halifax} \\
-3 & P & \emph{Buenos Aires} \\
-2 & O & Greenland \\
-1 & N & Azores \\
\bottomrule
\end{tabular}
\end{minipage}
\begin{minipage}{0.4\textwidth}
\begin{tabular}{rcl}
\toprule
UTC & & Example \\
\midrule
+1 & A & \emph{Vienna}, France \\
+2 & B & \emph{Athens}, Greece\\
+3 & C & Iraq, Kuwait, Qatar\\
+4 & D & \emph{Moscow}, Russia\\
+5 & E & Pakistan, Kazakhstan\\
+6 & F & Bangladesh \\
+7 & G & Thailand \\
+8 & H & \emph{Bejing}, China \\
+9 & I & \emph{Tokyo}, Japan \\
+10 & K & \emph{Brisbane}, Australia \\
+11 & L & \emph{Sydney}, Australia \\
+12 & M & \emph{Wellington}, New Zealand \\
\bottomrule
\end{tabular}
\end{minipage}

\bigskip\noindent\marginpar{milToday} The \textbf{milToday} command sets the current date in NATO format

\noindent\marginpar{milDate} The \textbf{milDate} command sets the date consisting of day, month, year. It can be specified in the format \emph{dd-mm-yyyy} or in the ISO format \emph{yyyy-mm-dd}. For clear distinction, years should be specified with four digits. All non-numeric characters can be used as delimiter. For the current day the macro \verb+\today+ can be used
The date is run through a simple validity check, where days range from 1 to 31 and months from 01 to 12.

\par\medskip
\begin{tabular}{ll}
\verb+\milDate{08.12.1980}+ & \milDate{08.12.1980} \\
\verb+\milDate{10/05/1994}+ & \milDate{10/05/1994} \\ 	% Nelson Mandela sworn in as first black president
\verb+\milDate{4-7-97}+ 	& \milDate{4-7-97} \\		% Mars Pathfinder lands on Mars
\verb+\milDate{2008.11.4}+ 	& \milDate{2008.11.4} \\	% Barack Obama elected
\verb+\milDate{\today}+	  & \milDate{\today} \\		% this day
\end{tabular}

\medskip\noindent\marginpar{milDatetime} The \textbf{milDatetime} command allows to set date together with time. For the current day the macro \verb+\today+ can be used.
The date and time are run through simple validity checks, where days range from 1 to 31,months from 1 to 12, hours from 0 to 23 and minutes from 0 to 59.

\par\medskip
\begin{tabular}{ll}
\verb+\milDatetime{02.09.1966}{01:00}+		& \milDatetime{02.09.1966}{01:00} \\	% Great fire of London starts
\verb+\milDatetime{26/4/1986}{01:24}+		& \milDatetime{26/4/1986}{01:24} \\ 	% Explosion at Chernobyl
\verb+\milDatetime{1972-06-17}[R]{01:47}+	& \milDatetime{1972-06-17}[R]{01:47} \\	% Watergate burglars arrested
\verb+\milDatetime{1963/8/8}{03:15}+		& \milDatetime{1963/8/8}{03:15} \\		% Great Train Robbery
\verb+\milDatetime{\today}{05:00}+			& \milDatetime{\today}{05:00} \\	

\end{tabular}
\end{document}